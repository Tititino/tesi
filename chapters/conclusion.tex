\chapter{Conclusion}\label{chapter:conclusion}
The purpose of the original paper \cite{HarlandPym} was to show a general way to handle splitting which could be adapted to a series of different logics.
In \S\ref{chapter:testing} we show that, after modifying the original calculus using techniques from proof theory (i.e. focusing \cite{Focusing} and normalization), this usage of constraints is an effective way of handling multiplicatives during linear logic proof search.
Furthermore in \S\ref{chapter:implementation}, our Prolog implementation demonstrates how this method can be fairly easily integrated into a standard focusing prover without needing excessive modification, in part thanks to Prolog's elegant interface for CLP.

There are many ways in which our work may be expanded in the future, just to cite a few:
\begin{itemize}
	\item our prover as of right now handles exponentials rather poorly (see Section \ref{sec:benchmarking}), research could be made to devise alternative heuristics or a more refined use of constraints;
	\item a constraint calculus for intuitionistic linear logic (ILL) may be defined and a prover similar to ours implemented; the same could be said about the logic of bunched implications (BI) using the calculus described in \cite{HarlandPym};
	\item more interesting uses of constraints may be explored, perhaps using disjunction on the axioms such that
		$$ \vdash \af{a}{e_1}, \af{a}{e_2}, \af{\llnot{a}}{e_3} $$
	generates the constraints
		$$ ((\used{e_1} \wedge \avail{e_2}) \vee (\avail{e_1} \wedge \used{e_2})) \wedge \used{e_3} $$
\end{itemize}
