\chapter{State of the art}
Most forward provers for classic linear logic use some combination of focusing and normalization to structure their proofs, with the notable exception of llprover not using normalization.	% credo
We confront our prover with two other provers: llprover (1997, %cit
) and APLL (circa 2019, %cit
).

Usually the splitting is handled in two ways: trying every partition possible, or using something called the method of input/output %cit
. The latter tries to do one branch of the proof of a multiplicative, and then feeds the remaining formulae in the sequent of the other branch.

We now give a deeper look at the provers we confront with.

\section{APLL}\label{sec:apll}
APLL is the underlying prover of click\&collect. % cit
It provides 4 different searches -- forward and backwards for classic and intuitionistic linear logic. 
We will focus on the backwards algorithm for classic linear logic.


The program is written in OCaML and implements a standard focused proof search on normalized formulae as seen in \cite{LiangMiller}.
In this section we will illustrate two noteworthy charateristics of its implementation:
\begin{itemize}
	\item Sequent splitting when encountering a tensor is done by generating all the numbers up to $2^{|\Delta|}$ -- where $\Delta$ is the sequent -- and using the bit representation of those to create the two subsets.
		This can be seen in the function \texttt{split\_list}, which in turn calls \texttt{split\_list\_aux}
		\begin{minted}{ocaml}
let rec split_list_aux (acc1, acc2) l k = match l with
  | [] -> acc1, acc2
  | hd :: tl -> 
      if k mod 2 = 0 
      then split_list_aux (acc1, hd :: acc2) tl (k / 2)
      else split_list_aux (hd :: acc1, acc2) tl (k / 2)
		\end{minted}
		where the argument \texttt{k} is the number that determines the decomposition of the sequent.
		This function is called recursively when a tensor is encountered during proof search, starting at $ k = 2^{|\Delta|}$ and decreasing by one at each iteration
		\begin{minted}{ocaml}
(* ... *)
| Tensor (g, h) ->
  let rec split_gamma k = 
    if k = -1 then None
    else
      let gamma1, gamma2 = split_list gamma k in
        try
	  (* ... *)
        with NoValue ->
          split_gamma (k - 1) 
  in
    let k = fast_exp_2 (List.length gamma) - 1 in
      (* ... *)
		\end{minted}

		As we will see in \ref{sec:benchmarking} this implementation choice will result in a degradation of performance on formulae with a high number of multiplicatives.
\end{itemize}

\section{\texttt{llprover}}
\texttt{llprover} is a prover by Naoyuki Tamura.
Where APLL had different provers for implicative and classical linear logic, this prover encodes all the rules as the same predicate \texttt{rule/6}, using the first argument as a selector for the system.
Using classical logic as the system uses all the rules, included the ones for implicative linear logic.
For this reason the prover does not implement normalization to NNF before proof search; instead the option is given to transform the two-sided proof to a one-sided one.

Tensor splitting is implemented similarly to APLL by trying every possible partition
\begin{minted}[linenos]{prolog}
rule([ill,0], no,  r(*), S, [S1, S2], [r(N),r(N1),r(N2)]) :-
  match(S,  ([X]-->[Y1,[A*B],Y2])),
  merge(X1, X2, X),
  merge(Y11, Y12, Y1),
  merge(Y21, Y22, Y2),
  match(S1, ([X1]-->[Y11,[A],Y21])),
  match(S2, ([X2]-->[Y12,[B],Y22])),
  length(Y1, N), length(Y11, N1), length(Y12, N2).
\end{minted}
Here \texttt{merge/3} (lines 3, 4 and 5) -- called with the only the third argument bound -- generates all possible lists that when merged together return the original sequents.

Another particular characteristic of llprover is that it uses a local bound with iterative deepening: llprover will try to prove the formula with bounds $1, 2, \dots$ up to the maximum specified.
This guarantees finding the simplest proof, at the expense of the overall speed.

