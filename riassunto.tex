\documentclass[a4paper, 12pt, italian]{article}

\usepackage[italian]{babel}
\usepackage[margin=2.5cm]{geometry}
\usepackage{setspace}
\usepackage[pdfa]{hyperref}
\usepackage[a-1a]{pdfx}

\begin{document}

\section{Ente presso cui è stato svolto il lavoro di stage}
Il lavoro di tirocinio è stato di tipologia interna presso il dipartimento di Informatica dell'Università degli Studi di Milano, sotto la supervisione del docente prof. Alberto Momigliano e il docente prof. Camillo Fiorentini.

\section{Contesto iniziale}
Nell'ambito dei dimostratori automatici bottom-up basati sul calcolo dei sequenti per la logica lineare, una delle principali fonti di complessità è un'operazione chiamata splitting.
Questa richiede che ripetutamente durante la computazione della dimostrazione sia necessario partizionare le conclusioni o le assunzioni, portando ad una esplosione combinatoria che si aggiunge all'inerente complessità del proof searching.

Il lavoro di tirocinio inizia da un articolo del 2001 di J. Harland e D. Pym \cite{HarlandPym} dove si propone un metodo alternativo per affrontare lo splitting affidandosi a vincoli booleani.
I vincoli vengono generati in modo tale che se questi sono soddisfacibili, allora la dimostrazione è corretta;
in tal modo i multiset di formule non vengono mai effettivamente partizionati, e la complessità viene spostata dalla generazione dei sottoinsiemi corretti, alla ricerca di un assegnamento per i vincoli booleani.

\section{Obiettivi del lavoro}
L'obiettivo principale del lavoro è stato implementare un prover basato sul calcolo di cui sopra e, una volta fatto questo, valutarne la efficacia anche rispetto ad altri prover esistenti.

\section{Descrizione del lavoro svolto}
Inizialmente è stato implementato un prover basato direttamente sul calcolo di \cite{HarlandPym}, ma questo si è rivelato inefficiente.
Successivamente sono state applicate alcune modifiche ormai assodate nell'ambito del theorem proving quali la normalizzazione e il focusing \cite{Focusing}.
Di questo nuovo calcolo è stata prima dimostrata la correttezza, esibendo una traduzione verso il calcolo triadico di \cite{Focusing}, poi ne è stata fatta una implementazione in Prolog.
Infine il programma è stato confrontato con due altri prover simili: llprover \cite{llprover} e APLL \cite{APLL}.

\section{Tecnologie coinvolte}
Sono state coinvolte le seguenti tecnologie:
\begin{itemize}
	\item il linguaggio SWI-Prolog per la scrittura del prover, con particolare enfasi sulle sue librerie per il constraint logic programming (CLP);
	\item il linguaggio OCaML per la scrittura di un generatore di formule randomiche;
	\item il linguaggio Python e Jupyter Notebook per l'infrastruttura necessaria per comparare diversi prover;
	\item il linguaggio Nix (in particolar modo l'estensione dei flake) e bash per l'automazione dei processi di compilazione e la generazione di ambienti di sviluppo riproducibili.
\end{itemize}

\section{Competenze e risultati raggiunti}
I benchmark hanno evidenziato come l'utilizzo dei vincoli booleani, assieme al focusing e alla normalizzazione, permettano di ottenere risultati competitivi nell’ambito della logica lineare moltiplicativa.

Il lavoro di tirocinio mi ha dato l'opportunità di avvicinarmi all’ambito della dimostrazione automatica, e a familiarizzarmi con la notazione del calcolo dei sequenti.
Inoltre l’utilizzo di Prolog per l'implementazione mi ha permesso di esplorare diversi ambiti della programmazione logica, ad esempio le CFG e CLP.

Una delle problematiche principali si è rivelata essere la carenza di test senza esponenziali -- un particolare tipo di connettivo della logica lineare; infatti la quasi totalità dei dataset di formule di logica lineare è costituito da traduzioni di teoremi intuizionisti o di reti di Petri, entrambi casi caratterizzati da un alto numero di esponenziali.
Visto che il nostro obiettivo principale era di valutare il prover nel caso moltiplicativo, è stato necessario trovare fonti alternative di formule.
A questo scopo abbiamo modificato un generatore randomico di formule affinché producesse casi senza esponenziali.

\newpage
\bibliographystyle{plain}
\bibliography{refs}

\end{document}
