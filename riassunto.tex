\documentclass[a4paper, 12pt, italian]{extarticle}

\usepackage{babel}
\usepackage[margin=2.5cm]{geometry}
\usepackage{parskip}

\usepackage[shortlabels]{enumitem}

\usepackage[a-1b]{pdfx}

% \usepackage{hyperref}
% \hypersetup{
%     colorlinks=true,
%     urlcolor=blue,
%     linkcolor=blue,
% }

\usepackage{biblatex}
\addbibresource{refs.bib}

\begin{document}

Questo documento deve descrivere per brevi capitoletti il lavoro svolto dallo studente durante lo stage.
Seguendo il presente formato si raccomanda di contenere la descrizione in al massimo quattro pagine (ma
non meno di una).

\section{Ente presso cui è stato svolto il lavoro di stage}
Descrizione dell’azienda/ente/Laboratorio e dell’ambiente di lavoro in cui si \`e andati ad operare.

\section{Contesto iniziale}
Inquadramento dello stage, situazione esistente.

\section{Obiettivi del lavoro}
Cosa si voleva raggiungere inizialmente.

\section{Descrizione del lavoro svolto}
Cosa \`e stato effettivamente fatto.

\section{Tecnologie coinvolte}
Quali tecnologie sono state utilizzate e apprese.

\section{Competenze e risultati raggiunti}
\begin{itemize}
	\item Quali risultati sono stati raggiunti rispetto agli obiettivi iniziali?
	\item Quali insegnamenti si possono trarre dall’esperienza effettuata?
	\item Quali i problemi incontrati? Quali risolti e quali no? Perch\'e?
\end{itemize}

\newpage

\section{\bibname}

\nocite{*}
\printbibliography[heading=none]

\end{document}
