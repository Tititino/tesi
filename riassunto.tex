\documentclass[a4paper, 12pt, italian]{extarticle}

\usepackage{babel}
\usepackage[margin=2.5cm]{geometry}
\usepackage{parskip}

\usepackage[shortlabels]{enumitem}

\usepackage[a-1b]{pdfx}

% \usepackage{hyperref}
% \hypersetup{
%     colorlinks=true,
%     urlcolor=blue,
%     linkcolor=blue,
% }

\usepackage{biblatex}
\addbibresource{refs.bib}

\begin{document}

Questo documento deve descrivere per brevi capitoletti il lavoro svolto dallo studente durante lo stage.
Seguendo il presente formato si raccomanda di contenere la descrizione in al massimo quattro pagine (ma
non meno di una).

\section{Ente presso cui è stato svolto il lavoro di stage}
Il lavoro di tirocinio è stato di tipologia interna presso il dipartimento di Informatica dell'Università degli Studi di Milano, sotto la supervisione del docente Alberto Momigliano e il docente Camillo Fiorentini.

\section{Contesto iniziale}
Il lavoro di tirocinio si concentra sulla definizione di un calcolo per la risoluzione automatica di teoremi in logica lineare utilizzando constraint booleani; l'implementazione di questo calcolo; e il suo confronto con altri prover già esistenti.

\section{Obiettivi del lavoro}
Cosa si voleva raggiungere inizialmente.

\section{Descrizione del lavoro svolto}
Cosa \`e stato effettivamente fatto.

\section{Tecnologie coinvolte}
Sono state coinvolte le seguenti tecnologie:
\begin{itemize}
	\item Linguaggio Prolog per la scrittura del prover, OCaML per la scrittura di un generatore di formule;
	\item Linguaggio Python per l'infrastruttura necessaria per comparare diversi prover;
	\item Linguaggio Nix (flakes) e bash per l'automatizzazione dei processi di compilazione e la generazione di ambienti di sviluppo riproducibili.
\end{itemize}

\section{Competenze e risultati raggiunti}
\begin{itemize}
	\item Quali risultati sono stati raggiunti rispetto agli obiettivi iniziali?
	\item Quali insegnamenti si possono trarre dall’esperienza effettuata?
	\item Quali i problemi incontrati? Quali risolti e quali no? Perch\'e?
\end{itemize}

\newpage

\section{\bibname}

\nocite{*}
\printbibliography[heading=none]

\end{document}
