\documentclass[a4paper, 12pt, tesi, english]{report}
\usepackage{tesi}

\begin{document}

\chapter{The focused calculus}

\begin{define}
	Given a boolean variable $x$, we use the following notation
	\begin{itemize}
		\item $\used{x}$ to state that the formula associated to the boolean variable $x$ gets consumed, this corresponds to saying the constraint $x$ is true
		\item $\notUsed{x}$ to state that the formula associated to the boolean variable $x$ does not get consumed, this corresponds to saying the constraint $x$ is not true
	\end{itemize}
	We then extends these predicates to sequents
	\begin{align*}
		\used{\Delta} &= \{ \used{e_2} \mid e_2 \in \text{exp}(\Delta) \} \\
		\notUsed{\Delta} &= \{ \notUsed{e_2} \mid e_2 \in \text{exp}(\Delta) \}
	\end{align*}
\end{define}

\begin{define}
	$\Pi$ represents a set of contraints.
	Furthermore if the set $\Pi$ is satisfiable we write
	$$ \sat{\Pi} \;\Longleftrightarrow\; \bigwedge_{\pi \in \Pi} \pi \text{ is \textit{sat}} $$
\end{define}

\begin{define}
	An annotated formula is a formula associated to a boolean expression.
	This is represented as
	$$ \text{af}(\phi, x) $$
	We will usually use $\phi$ to indicate formulas, and $x$ and $e$ to indicate respectively boolean variables and boolean expressions.
\end{define}

\begin{define}
	Given a sequent of annotated formulae $\Delta$ we define the operation of splitting it as
	$$ \text{split}(\Delta) \mapsto (\Delta^L, \Delta^R) $$
	where, given a set $X$ of new variable names for each formula in $\Delta$
	\begin{align*}
		\Delta^L &= \{ \af{\phi_i}{x_i \wedge e_i} \mid i \in \{1, \dots, n\}\} \\
		\Delta^R &= \{ \af{\phi_i}{\varNot{x_i} \wedge e_i} \mid i \in \{1, \dots, n\}\} \\
	\end{align*}
	with $n$ the cardinality of $\Delta$, and $\phi_i$ and $e_i$ respectively the formula and the variable of the $i$-eth annotated formula in $\Delta$ using an arbitrary order.
\end{define}

\begin{figure}[H]
	\centering
	\begin{subfigure}{\textwidth}
		\centering
		\begin{tblr}{ colspec = { cc }, rows = {abovesep=10pt, belowsep=10pt}}
			\SetCell[c=2]{c} {\small
			\AxiomC{$\async{\used{e}, \Pi}{\Psi}{\Delta}{\af{\phi_1}{e}, \af{\phi_2}{e}, \Phi}$}
			\LeftLabel{$[\llpar]$}
			\UnaryInfC{$\async{\Pi}{\Psi}{\Delta}{\af{\phi_1 \llpar \phi_2}{e}, \Phi}$}
			\DisplayProof} \\
			{\small
			\AxiomC{$\async{\used{e}, \Pi}{\Psi}{\Delta}{\Phi}$}
			\LeftLabel{$[\llbot]$}
			\UnaryInfC{$\async{\Pi}{\Psi}{\Delta}{\af{\llbot}{e}, \Phi}$}
			\DisplayProof}
			&
			{\small
			\AxiomC{}
			\LeftLabel{$[\lltop]$}
			\UnaryInfC{$\async{\Pi}{\Psi}{\Delta}{\af{\lltop}{-}, \Phi}$}
			\DisplayProof
			}
			\\
			\SetCell[c=2]{c} {\small
			\AxiomC{$\async{\used{e}, \Pi}{\Psi}{\Delta}{\af{\phi_2}{e}, \Phi}$}
			\AxiomC{$\async{\used{e}, \Pi}{\Psi}{\Delta}{\af{\phi_1}{e}, \Phi}$}
			\LeftLabel{$[\llwith]$}
			\BinaryInfC{$\async{\Pi}{\Psi}{\Delta}{\af{\phi_1 \llwith \phi_2}{e}, \Phi}$}
			\DisplayProof}
			\\
			\SetCell[c=2]{c} {\small
			\AxiomC{$\async{\Pi}{\Psi, \phi}{\Delta}{\Phi}$}
			\LeftLabel{$[?]$}
			\UnaryInfC{$\async{\Pi}{\Psi}{\Delta}{\af{\llwn{\phi}}{-}, \Phi}$}
			\DisplayProof} 
			\\
			\SetCell[c=2]{c} {\small
			\AxiomC{$\isNotAsy{\phi}$}
			\AxiomC{$\async{\Pi}{\Psi}{\Delta, \af{\phi}{e}}{\Phi}$}
			\LeftLabel{$[R\!\Uparrow]$}
			\BinaryInfC{$\async{\Pi}{\Psi}{\Delta}{\af{\phi}{e}, \Phi}$}
			\DisplayProof
			}
		\end{tblr}
		\caption{Asynchronous rules}
	\end{subfigure}

	\begin{subfigure}{\textwidth}
		\centering
		\begin{tblr}{colspec = { cc }, rows = {abovesep=10pt, belowsep=10pt}}
			\SetCell[c=2]{c} {\small
			\AxiomC{$\focus{\used{e}, \Pi}{\Psi}{\Delta^L}{\text{af}(\phi_1, e)}$}
			\AxiomC{$\focus{\used{e}, \Pi}{\Psi}{\Delta^R}{\text{af}(\phi_2, e)}$}
			\LeftLabel{$[\llten]$}
			\BinaryInfC{$\focus{\Pi}{\Psi}{\Delta}{\text{af}(\phi_1 \llten \phi_2, e)}$}
			\DisplayProof}
			\\ 
			{\small
			\AxiomC{$\focus{\used{e}, \Pi}{\Psi}{\Delta}{\af{\phi_1}{e}} $}
			\LeftLabel{$[\llplus_L]$}
			\UnaryInfC{$\focus{\Pi}{\Psi}{\Delta}{\af{\phi_1 \llplus \phi_2}{e}}$}
			\DisplayProof}
			&
			{\small
			\AxiomC{$\focus{\used{e}, \Pi}{\Psi}{\Delta}{\af{\phi_2}{e}}$}
			\LeftLabel{$[\llplus_R]$}
			\UnaryInfC{$\focus{\Pi}{\Psi}{\Delta}{\af{\phi_1 \llplus \phi_2}{e}}$}
			\DisplayProof}
			\\
			{\small
			\AxiomC{$\sat{ \used{e_1}, \notUsed{\Delta}, \Pi}$}
			\LeftLabel{$[1]$}
			\UnaryInfC{$\focus{\Pi}{\Psi}{\Delta}{\af{\llone}{e_1}}$}
			\DisplayProof} 
			&
			{\small
			\AxiomC{$\focus{\used{e_1}, \notUsed{\Delta}, \Pi}{\Psi}{\Delta}{\af{\phi}{e_1}}$}
			\LeftLabel{$[!]$}
			\UnaryInfC{$\focus{\Pi}{\Psi}{\Delta}{\af{\llbang{\phi}}{e_1}}$}
			\DisplayProof
			}
			\\
			\SetCell[c=2]{c} {\small
			\AxiomC{$\isAsy{\phi} \vee \isNegLit{\phi}$}
			\AxiomC{$\async{\Pi}{\Psi}{\Delta}{\af{\phi}{e}}$}
			\LeftLabel{$[R\!\Downarrow]$}
			\BinaryInfC{$\focus{\Pi}{\Psi}{\Delta}{\af{\phi}{e}}$}
			\DisplayProof
			}
		\end{tblr}
		\caption{Synchronous rules}
	\end{subfigure}

	\begin{subfigure}{\textwidth}
		\centering
		\begin{tblr}{colspec = { cc }, rows = {abovesep=10pt, belowsep=10pt}}
			{\small
			\AxiomC{$ \sat{\used{e_1}, \used{e_2}, \notUsed{\Delta}, \Pi}$}
			\LeftLabel{$[I_1]$}
			\UnaryInfC{$\focus{\Pi}{\Psi}{\Delta, \af{\phi}{e_2}}{\af{\llnot{\phi}}{e_1}}$}
			\DisplayProof}
			&
			{\small
			\AxiomC{$\isNotNegLit{\phi}$}
			\AxiomC{$\focus{\Pi}{\Psi}{\Delta}{\af{\phi}{e}}$}
			\LeftLabel{$[D_1]$}
			\BinaryInfC{$\async{\Pi}{\Psi}{\Delta, \af{\phi}{e}}{.}$}
			\DisplayProof}
			\\
			{\small
			\AxiomC{$ \sat{\used{e_1}, \notUsed{\Delta}, \Pi}$}
			\LeftLabel{$[I_2]$}
			\UnaryInfC{$\focus{\Pi}{\Psi, \phi}{\Delta}{\af{\llnot{\phi}}{e_1}}$}
			\DisplayProof}
			&
			{\small
			\AxiomC{$\isNotNegLit{\phi}$}
			\AxiomC{$\focus{e\;new, \Pi}{\Psi}{\Delta}{\af{\phi}{e}} $}
			\LeftLabel{$[D_2]$}
			\BinaryInfC{$\async{\Pi}{\Psi, \phi}{\Delta}{.}$}
			\DisplayProof}
		\end{tblr}
		\caption{Identity and decide rules}
	\end{subfigure}

\end{figure}

\end{document}
